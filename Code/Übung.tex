\documentclass[fontsize=10pt,a4paper]{scrartcl}

\usepackage{ucs}
\usepackage[utf8x]{inputenc}
\usepackage[T1]{fontenc}
\usepackage[ngerman]{babel}
\usepackage{amsmath,amssymb,amstext}
\usepackage{siunitx}  
\usepackage{eurosym}
\usepackage{subfig}
\usepackage{geometry}
\geometry{
	left=3cm,
	right=3cm,
	top=1.5cm,
	bottom=3cm,
}

\title{CG Übung I}
\author{Lukas Sabatschus}
\date{23.4.2020}
\makeatletter
\def\@seccntformat#1{%
	\expandafter\ifx\csname c@#1\endcsname\c@section
	Aufgabe \thesection:
	\else
	\csname the#1\endcsname\quad
	\fi}
\makeatother
\begin{document}
\maketitle
%\tableofcontents \newpage

\section{Theoretischer Aufgabenteil}
\subsection{a}
Für 3 Punkte $\mathbf{p}_0,\mathbf{p}_1,\mathbf{p}_2\in \mathbb{R}^2$
eines Dreiecks $\Delta(\mathbf{p}_0,\mathbf{p}_1,\mathbf{p}_2)$
soll gezeigt werden, das aus
\begin{align*}
	\lambda_0\mathbf{p}_0+\lambda_1\mathbf{p}_1+\lambda_2\mathbf{p}_2&=\mathbf{p}\\
	\lambda_0+\lambda_1+\lambda_2&=1
\end{align*}
folgende $\lambda$s folgen:
\begin{align*}
	\lambda_0 & =\frac{A(\Delta(\mathbf{p},\mathbf{p}_1,\mathbf{p}_2))}{A(\Delta(\mathbf{p}_0,\mathbf{p}_1,\mathbf{p}_2))} \\
	\lambda_1 & =\frac{A(\Delta(\mathbf{p}_0,\mathbf{p},\mathbf{p}_2))}{A(\Delta(\mathbf{p}_0,\mathbf{p}_1,\mathbf{p}_2))} \\
	\lambda_2 & =\frac{A(\Delta(\mathbf{p}_0,\mathbf{p}_1,\mathbf{p}))}{A(\Delta(\mathbf{p}_0,\mathbf{p}_1,\mathbf{p}_2))} \\
\end{align*}

Dafür lösen wir folgendes LGS:
\begin{align}
	\lambda_0\mathbf{p}_{0x}+\lambda_1\mathbf{p}_{1x}+\lambda_2\mathbf{p}_{2x}&=\mathbf{p}_x\\
	\lambda_0\mathbf{p}_{0y}+\lambda_1\mathbf{p}_{1y}+\lambda_2\mathbf{p}_{2y}&=\mathbf{p}_y\\
	\lambda_0+\lambda_1+\lambda_2&=1
\end{align}

\begin{align}
(3) umgeformt:\space \lambda_0&=1-\lambda_1-\lambda_2\\
(4) in (2):\mathbf{p}_y&=\space (1-\lambda_1-\lambda_2)\mathbf{p}_0y+\lambda_1\mathbf{p}_1y+\lambda_2\mathbf{p}_2y\\
\lambda_2&=\frac{-\lambda_1\mathbf{p}_{0y}+\lambda_1\mathbf{p}_{1y}-\mathbf{p}_{y}+\mathbf{p}_{0y}}{\mathbf{p}_{0y}-\lambda_1\mathbf{p}_{2y}}\\
(3)\&(6)in(1): \textnormal{[zu lang für meine \LaTeX-skills]}\\
\textnormal{Nach }\lambda_0\textnormal{ umgeformt:}\lambda_0&=\frac{\mathbf{p}_y(\mathbf{p}_{0x}-\mathbf{p}_{2x})+\mathbf{p}_{x}(\mathbf{p}_{0y}-\mathbf{p}_{2y})
	-\mathbf{p}_{0y}\mathbf{p}_{2x}+\mathbf{p}_{0x}\mathbf{p}_{2y}}{\mathbf{p}_{0y}(\mathbf{p}_{1x}-\mathbf{p}_{2x})+\mathbf{p}_{0x}(\mathbf{p}_{0x}(\mathbf{p}_{2y}-\mathbf{p}_{1y}+\mathbf{p}_{1y}\mathbf{p}_{2x}-\mathbf{p}_{1x}\mathbf{p}_{2y}))}\\
	\Rightarrow \lambda_0 & =\frac{\vert\left(\mathbf{p}_{1x}-\mathbf{p}_{x}\right)-\left(\mathbf{p}_{2y}-\mathbf{p}_{y}\right)\vert}{\vert\left(\mathbf{p}_{1x}-\mathbf{p}_{0x}\right)-\left(\mathbf{p}_{2y}-\mathbf{p}_{0y}\right)\vert} \\
	\Rightarrow \lambda_0 & =\frac{A(\Delta(\mathbf{p},\mathbf{p}_1,\mathbf{p}_2))}{A(\Delta(\mathbf{p}_0,\mathbf{p}_1,\mathbf{p}_2))} \\
\end{align}
Wobei $A(\Delta(a,b,c))$ die Fläche des durch $\mathbf{a},\mathbf{b},\mathbf{c}\in\mathbb{R}^2$ aufgespannten Dreiecks bezeichnet.\\
Analoges unformen für $\lambda_1$\&$\lambda_2$ ergibt die entsprechenden Frormeln von oben.

\end{document}











